\setlength{\tabcolsep}{6pt}
\begin{tabu} to \linewidth {X[-2.5,c, m]|X[c,m]}
  $\displaystyle Har(n) = \sum_{i=1}^n \frac1n$ & $\displaystyle Number(u) =
  \sum_{i=1}^n u_i \cdot 2^{i-1}$ \\ \hline
  \onecol
  {Let $U$ be an integer-valued problem, and let $A$ be an algorithm that solves
   $U$. $A$ is a {\bf pseudo-polynomial-time algorithm for $U$} if there exists
   a polynomial $p(a,b)$ s.t. $Time_A(x) = \mathcal{O}(p(|x|,
   Max\text{-}Int(x)))$ for every $x \in U$.} \\ \hline
  \onecol
  {Let $U$ be an integer-valued problem, and let $h : \mathcal{N} \rightarrow
   \mathcal{N}$ be nondecreasing. the {\bf $h$-value-bounded subproblem of $U$,
   $Value(h)-U$}, is the problem obtained from $U$ by restricting the set of all
   input instances of $U$ to the set of input instances $x$ with
   $Max\text{-}Int(x) \leq h(|x|)$.} \\ \hline
   \bf ~~strongly \newline NP-hard & An integer-valued problem $U$ is {\bf
  strongly NP-hard} if there exists a polynomial $p$ such that
  $Value(p)\text{-}U$ is NP-hard.\\ \hline
  \onecol{\centering \bf Parameterized Complexity} \\ \hline
  \onecol{{\bf A parameterization of $\mathbf{U}$} is any function
  $\mathbf{Par: L} \rightarrow \mathbb{N}$ s.t.
  \begin{enumeratex}
  \item $\mathbf{Par}$ is polynomial-time computable
  \item for infinitely many $k \in \mathbb{N}$, the
  {\bf \textit{k}-fixed-parameter set} \newline
    $\mathbf{Set_U(k)} = \{x \in L \mid Par(x) = k\}$ is an infinite set.
  \end{enumeratex}} \\ \hline
  \onecol{$A$ is a {\bf \textit{Par}-parameterized polynomial-time algorithm
  for \textit{U}} if
  \begin{enumeratex}
  \item $A$ solves $U$
  \item there exists a poly. $p$ and a fn $f : \mathbb{N} \rightarrow
  \mathbb{N}$ s.t., $\forall x \in L$ \newline
  $Time_A(x) \leq f(Par(x)) \cdot p(|x|)$
  \end{enumeratex}}\\ \hline
  \onecol{$U$ is a {\bf fixed-parameter-tractable according to \textit{Par}},
  if there is a Par-parameterized polynomial-time algo. for $U$.}\\ \hline
  \onecol{\centering \bf Local Search} \\ \hline
  \bf {\footnotesize neighbourhood} \newline on $\M(x)$ &
  $\forall x \in L_I$, a {\bf neighbourhood on $\M(x)$} is any
  $f_x : \M(x) \rightarrow Pot(\M(x))$ such that\\ \cline{1-1}
  \onecol{%
  \begin{enumeratex}\vspace{-\medskipamount}
  \item $\alpha \in f_x(\alpha)$ for every $\alpha \in \M$
  \item if $\beta \in f_x(\alpha)$ for some $\alpha \in \M$, then
  $\alpha \in f_x(\beta)$
  \item for all $\alpha, \beta \in \M(x)$, there exists $k \in \N^+$ and
  $\gamma_1$, .., $\gamma_k \in \M(x)$ s.t. $\gamma_1 \in
  f_x(\alpha), \gamma_{i+1} \in f_x(\gamma_i)$ for $i = 1,.., k-1$ and
  $\beta \in f_x(\gamma_k)$
  \end{enumeratex}
  If $\alpha \in f_x(\beta)$ for some $\alpha, \beta \in \M(x)$, then $\alpha$
  and $\beta$ are {\bf neighbours in $\M(x)$}. The set $f_x(\alpha)$ is the {\bf
  neighbourhood of the feasible solution $\alpha$ in $\M(x)$.} The undirected
  graph \newline
  $\mathbf{G}_{\M(x),f_x} = \left(\M(x), \{\{\alpha, \beta\} \mid \alpha \in
  f_x(\beta), \alpha \neq \beta, \beta \in \M(x)\}\right)$ \newline
  is the {\bf neighbourhood graph of $\M(x)$ according to the neighbourhood
  $f_x$}. \vspace{1mm} \newline
  Let, for every $x \in L_I$, $f_x$ be a neighbourhood on $\M(x)$. The fn
  $f : \cup_{x\in L_I}(\{x\}\times \M(x)) \rightarrow \cup_{x\in L_I} Pot(\M(x))$
  with the property $f(x, \alpha) = f_x(\alpha)$ for every $x \in L_I$ and every
  $\alpha \in \M(x)$ is called a {\bf neighbourhood for \textit{U}}}\\ \hline
\end{tabu}
