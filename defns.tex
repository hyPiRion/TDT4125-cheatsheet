\setlength{\tabcolsep}{6pt}
\begin{tabu} to \linewidth {>{\bfseries}X[-2.5, c, m] | X[l,m]}
  alphabet \norm{($\Sigma$)} & Any non-empty finite set.\\ \hline
  symbol {\it of $\Sigma$} & An element of an alphabet.
  $\Sigma$\\ \hline
  word {\it over $\Sigma$} & Any finite sequence of symbols of $\Sigma$.
  \\ \hline
  $\lambda$ & {\bf the empty word}, $|\lambda| = 0$. \\ \hline
  $| w |$ & {\bf length of a word}, no. of symbols in $w$. \\ \hline
  $\#_a(w)$ & Number of occurences of $a$ in $w$. \\ \hline
  $\Sigma^*$ & Set of all words over the alphabet $\Sigma$. \newline
  $\Sigma^+ = \Sigma^* - \{\lambda\}$ \\ \hline
  $\Sigma^n$ & All words $w$ over $\Sigma^*$ where $| w | = n$.
  \\ \hline
  $vw$ or $v \cdot w$ & {\bf concatenation} of $v$ and $w$. \\ \hline
  prefix {\it of} $w$ & Any $v \in \Sigma^*$ s.t. $w = vu$ where $u \in \Sigma^*$.
  \\ \hline
  suffix {\it of} $w$ & Any $u \in \Sigma^*$ s.t. $w = vu$ where $v \in \Sigma^*$.
  \\ \hline
  subword {\it of} $w$ & Any $z \in \Sigma^*$ s.t. $w = uzv$ where  $v, u \in
  \Sigma^*$. \\ \hline
  language \norm{($L$)} & A set $L \subseteq \Sigma^*$ is a language {\it over}
  $\Sigma^*$. \\ \hline
  $L^C$ & The {\bf complement of the language $L$ according to $\Sigma$}:
  $L^C = \Sigma^* - L$. \\ \hline
  $L_1L_2 = L_1 \circ L_2$ & {\bf concatenation} {\it of $L_1$ and $L_2$}: \newline
  $L_1L_2 = \{uv \in \left(\Sigma_1 \cup \Sigma_2 \right)^* \mid u \in \Sigma_1,
  v \in \Sigma_2\}$ \\ \hline
  linear \newline ordering & if $\Sigma = \{s_1, \ldots, s_m\}$ then
  $s_1 < \ldots < s_m$ is a linear ordering of $\Sigma$. \\ \hline
  canonical \newline ordering & For all $u,v \in \Sigma^*$, $u < v$ iff $| u
  | < | v |$ {\bf or} $| u | = | v |$, $u = xs_iu^\prime$
  and $v = xs_{(i+k)}v^\prime$ \\ \hline
  \multicolumn{2}{>{\itshape}p{\dimexpr 1.4\tabucolX+\tabcolsep+\arrayrulewidth\relax}}
 {A {\bf decision problem} is a triple $(L, U, \Sigma)$ where $\Sigma$ is an
   alphabet and $L \subseteq U \subseteq \Sigma^*$. An algorithm A {\bf solves
     (decides)} the decision problem $(L, U, \Sigma)$ if, for every $x \in U$,
  \begin{enumerate} \itemsep1pt \parskip0pt \parsep0pt \vspace{-\medskipamount}
  \item $A(x) = 1$ if $x \in L$, and
  \item $A(x) = 0$ if $x \in U - L$ ($x \notin L$)
  \vspace{-\medskipamount} \end{enumerate}
  } \\ \hline
  \multicolumn{2}{>{\itshape}p{\dimexpr 1.4\tabucolX+\tabcolsep+\arrayrulewidth\relax}}
 {An {\bf optimization problem} is a 7-tuple $U = (\Sigma_I, \Sigma_O, L, L_I,
   \mathcal{M}, cost, goal)$ where
   \begin{enumerate} \itemsep1pt \parskip0pt \parsep0pt \vspace{-\medskipamount}
   \item $\Sigma_I$ is an alphabet, called the {\bf input alphabet} of $U$
   \item $\Sigma_O$ is an alphabet, called the {\bf output alphabet} of $U$
   \item $L \subseteq \Sigma_I^*$ is the {\bf language of feasible problem
     instances}
   \item $L_I \subseteq L$ is the {\bf language of the (actual) problem
     instances of $U$}
   \item $\mathcal{M}$ is a function from $L$ to $Pot(\Sigma_O^*)$ and, $\forall
     x \in L$, $\mathcal{M}(x)$ is called the set of {\bf feasible solutions}
     for $x$
   \item $cost$ is the {\bf cost function} that, $\forall (u, x)$, where
     $u \in \mathcal{M}(x)$ for some $x \in L$, assigns a positive real number
     $cost(u, x)$
   \item $goal \in \{min, max\}$
   \vspace{-\medskipamount} \end{enumerate}
   For every $x \in L_I$, a feasible solution $y \in \mathcal{M}(x)$ is called
   {\bf optimal for $x$ and $U$} if $cost(y, x) = goal\{cost(z,x) \mid z \in
   \mathcal{M}(x)\}$. \newline
   For an optimal solution $y \in \mathcal{M}(x)$, we denote $cost(x,y)$ by
   $Opt_U(x)$. $U$ is called a {\bf maximization problem} if $goal = max$,
   and $U$ is a {\bf minimization problem} if $goal = min$. \newline
   An algorithm A is {\bf consistent} for $U$ if, $\forall x \in L_I$, the
   output $A(x) \in \mathcal{M}(x)$. We say that an algorithm B {\bf solves} the
   optimization problem $U$ if
   \begin{enumerate} \itemsep1pt \parskip0pt \parsep0pt \vspace{-\medskipamount}
   \item $B$ is consistent for $U$
   \item $\forall x \in L_I$, $B(x)$ is an optimal solution for $x$ and $U$
   \vspace{-\medskipamount} \end{enumerate}} \\ \hline
 \multicolumn{2}{>{\itshape}p{\dimexpr 1.4\tabucolX+\tabcolsep+\arrayrulewidth\relax}}
 {Let $U_1 = (\Sigma_I, \Sigma_O, L, L_{I,1}, \mathcal{M}, cost,
 goal)$, and $ U_1 = (\Sigma_I, \Sigma_O, L, L_{I,2}, \mathcal{M}, cost, goal)$
 be two optimization problems. We say that $U_1$ is a {\bf subproblem} of $U_2$
 if $L_{I,1} \subseteq L_{I,2}$.}
\end{tabu}
