\setlength{\tabcolsep}{6pt}
\vspace{-4.7mm}
\begin{tabu} to \linewidth {X[-2.5,c]|X[c,m]}
  $f(n) = \mathcal{O}(g(n))$ & iff $\exists$ positive $c, n_0$ such that
  $0 \leq f(n) \leq cg(n)$ $\forall n \geq n_0$. \\ \hline
  $f(n) = \Omega(g(n))$ & iff $\exists$ positive $c, n_0$ such that
  $f(n) \geq cg(n) \geq 0$ $\forall n \geq n_0$. \\ \hline
  $f(n) = \Theta(g(n))$ & iff $f(n) = O(g(n))$ and $f(n) = \Omega(g(n))$.
  \\ \hline
  $f(n) = o(g(n))$ & \vspace{0.7\baselineskip} iff $\displaystyle \lim_{n \to \infty}
  f(n)/g(n) = 0$. \newline \vspace{-0.2\baselineskip} \\ \hline
  $\displaystyle Har(n) = \sum_{i=1}^n \frac1n$ & $\displaystyle Number(u) =
  \sum_{i=1}^n u_i \cdot 2^{i-1}$ \\ \hline
  $Pot(S) = \{Q \mid Q \subseteq S\}$ & {\bf TODO: put something here} \\ \hline
  \vspace{-\baselineskip} ~~~~$Time_A(x)$ \newline $Space_A(x)$ &
  Denotes the time/space complexity of the computation of $A$ on input $x \in
  \Sigma_I^*$ where $A: \Sigma_I^* \rightarrow \Sigma_O^*$ \\ \cline{1-1}
  \multicolumn{2}{p{\dimexpr 1.6\tabucolX+\tabcolsep+\arrayrulewidth\relax}}
  {{\bf (worst case) complexity of $\mathbf{A}$} is defined by
    $X_A(n) = max\{X_A(x) \mid x \in \Sigma_I^*\}$ , where $X$ is
    $Time$/$Space$.} \\ \hline
  $\mathbf{P}$ & \it $\{L = L(M) \mid M $ is a TM (algorithm) with
  $Time_M(n) \in \mathcal{O}(n^c), c \in \mathcal{N}^+ \}$ \\ \cline{1-1}
  \multicolumn{2}{p{\dimexpr 1.6\tabucolX+\tabcolsep+\arrayrulewidth\relax}}
  {A language (decision problem) is called {\bf tractable (practically
      solveable)} if $L \in \mathbf{P}$ and {\bf intractable} if
    $L \notin \mathbf{P}$.}\\ \hline
  \multicolumn{2}{p{\dimexpr 1.6\tabucolX+\tabcolsep+\arrayrulewidth\relax}}
  {Let $U$ be an integer-valued problem, and let $A$ be an algorithm that solves
    $U$. $A$ is a {\bf pseudo-polynomial-time algorithm for $U$} if there exists
    a polynomial $p(a,b)$ s.t. $Time_A(x) = \mathcal{O}(p(|x|,
    Max\text{-}Int(x)))$ for every $x \in U$.} \\ \hline
  \multicolumn{2}{p{\dimexpr 1.6\tabucolX+\tabcolsep+\arrayrulewidth\relax}}
  {Let $U$ be an integer-valued problem, and let $h : \mathcal{N} \rightarrow
    \mathcal{N}$ be nondecreasing. the {\bf $h$-value-bounded subproblem of $U$,
  $Value(h)-U$}, is the problem obtained from $U$ by restricting the set of all
    input instances of $U$ to the set of input instances $x$ with
    $Max\text{-}Int(x) \leq h(|x|)$.} \\ \hline
\end{tabu}
